\section{Los números complejos}
En el siglo XVI(1545) se empieza a trabajar con los numeros complejos, siendo su precursor G. Cardano.
No sin detractores entre ellos Descartes (1637) y A.De Morgan (1831).
Y por supuesto grandes adeptos como Euler, Gauss, Cauchy, Weierstrass y Riemman.

En 1837 Hamilton define \(\mathbb{C}^{}\) como pares ordenados de numeros reales \(\mathbb{R}^{2}=\{(a,b) | a,b \in \mathbb{R}^{}\}\) con las siguientes dos operaciones:

-Suma: \((a,b)+(c,d) = (a+c, b+d)\)
-Producto: \((a,b)*(c,d) = (ac-bd, ad+bc)\)

Estas operaciones son cerradas y con ellas se satisfacen las propiedades de cuerpo:

% Falta convertir en lista
1. La suma es asociativa
2. La suma tiene elemento neutro \((0,0)\).
3. Existe elemento opuesto para la suma. El opuesto de \((a,b)\) es \((-a,-b)\)
4. La suma es conmutativa.
5. La multiplicacion es asociativa.
6. La multiplicacion tiene elemento neutro (distinto del de la suma) \((1,0)\)
7. Elemento inverso para la multiplicacion. El inverso multiplicativo de \((a,b) \neq (0,0)\) es \((\frac{a}{a^2+b^2},\frac{-b}{a^2+b^2})\)
8. La multiplicacion es conmutativa.
9. Propiedad distributiva del producto respecto de la suma. \((a,b)((c,d)+(e,f))=(a,b)(c+e,d+f) = \)

Conclusión: \((\mathbb{R}^{2},+,*)\) es cuerpo conmutativo. Lo llamamos \(\mathbb{C}^{}\) y sus elemntos se llaman numeros complejos.

Veamos que los numeros reales son subcuerpo de \(\mathbb{C}^{}\).

\[ E= \{(a,0) | a \in \mathbb{R}^{}\} \subset \mathbb{C}^{}\]

Las operaciones suma y producto son cerradas en E:

-\((a,0) + (b,0) = (a+b, 0) \in E, a,b \in \mathbb{R}^{}\) 
-\((a,0)*(b,0) = (a*b, 0) \in E, a,b \in \mathbb{R}^{}\)

y el opuesto aditivo de \( (a,0) \in E\) es \( (-a,0) \in E\) asi como el inverso multiplicativo de  \( (a,0) \in E\) es \( (\frac{1}{a},0) \in E\).

Esto nos dice que E es subcuerpo de \(\mathbb{C}^{}\). Además, este cuerpo E es isomorfo a \(\mathbb{R}^{}\) (isomorfo en el sentido de cuerpo) mediante la identificacion \((a,0) \in E <-> a \in \mathbb{R}^{}\) % Falta cambiar flecha.

\section{Terminologia y nomeclatura}

- Los elementos de \(\mathbb{C}^{} <-> \mathbb{R}^{2}\) se llaman numeros complejos.
- Si \((a,b) \in \mathbb{C}^{} \) su parte real es a y su parte imaginaria es b
- Distinguimos dos elemntos de \(\mathbb{C}^{}\): \((1,0) = 1\) y \((0,1) = i\)

Mediante la identificacion \( E <-> \mathbb{R}^{}\). Tenemos que para \(x,y \in \mathbb{R}^{}\).

% Anotacion 1.

Luego \((x,y) = (x,0)+(0,y) = x + yi\).
De esta manera \(\mathbb{C}^{} = \{x+yi : (x,y) \in \mathbb{R}^{2}\}\).

Los numeros complejos se representan en \(\mathbb{R}^{2}\).
\(z = x+yi \in \mathbb{C}^{}\), \(Re(z) = x\) y \(Im(z) = y\).

% Insertar grafica 1.

- \( \mathbb{C}^{}\) no tiene orden. (A diferencia de \(\mathbb{R}^{}\))

- \(i^2 = (1,0)(0,1) = (0*0-1*1, 0*1+1*0)\).

Eston nos dice que i es solucion de la ecuacion
\begin{equation*}
  x^2+1=0
\end{equation*}
Otra solucion es -i.

- En \(\mathbb{R}^{}\), \(x^2+1=0 \) no tiene raices, pero en \(\mathbb{C}^{}\) si es factorizable como \(x^2+1= (x-i)(x+i)\)

- Lo anterior da otra forma algebraica de introducir \(\mathbb{C}^{}\)

En \(\mathbb{R}[x]\), anillo de los polinomios con coeficientes reales, consideramos el polinomio \(x^2+1=0\), el cual, al ser irreducible genera un ideal maximal en  \(\mathbb{R}[x]\).

\[ I = <x^2+1> = \{q(x)(x^2+1), q \in \mathbb{R}[x]\}\]

y por tanto, el cociente, \(\frac{\mathbb{R}[x]}{I} = \{p(x)+<x^2+1> : p \in \mathbb{R}[x]\}\) es cuerpo bajo una suma y multiplicacion inducidas por \(\mathbb{R}^{}[x]\).

Por el algoritmo de la division en \(\mathbb{R}^{}[x]\), todo polinomio p(x) se expresa de forma unica como

\[p(x) = q(x)(x^2+1) + a +bx\]

con \(a,b \in \mathbb{R}^{}\)

Esto nos dice que la clase de p(x) en \(\frac{\mathbb{R}[x]}{I}\) es:

%Anotacion 2

Esto nos permite establecer la siguiente identificacion:

%Anotacion 3.

Por ejemplo:
\[[x]^2=[x^2+1-1]=[x^2+1]-[1] = [0] - [1] = -[1]\]
\[(i^2 = -1)\]

\begin{definicion}
  Si \(z = a+ib \in \mathbb{C}^{}\), entonces definimos su conjugado como
  \[ z = a-ib\] %Falta poner la barra sobre z para indicar el conjugado

  Esta apliacion de conjugacion se consigue en \(\mathbb{R}^{2}\) con la siguiente aplicacion lineal:

  %Anotacion 4

  Geometricamente, z y z son simetricas con respecto al eje real.

  %Grafica 2.

  Propiedades:

  1. 0 = 0
  2. 1 = 1
  3. z+w = z+w
  4. z*w = z*w % Demostracion de esto en anotacion 5.
  5. Se trata de una involucion: z (dos veces conjugado ) = z
  6. Si \(z = a+ib \in \mathbb{C}^{}\), se tiene que
  \[ Re(z) = \frac{z+z_}{2}\]
  \[Im(z) = \frac{z-z_}{2}\]
\end{definicion}

\section{\(\mathbb{C}^{}\) como espacio vectorial}

-Al estar \(\mathbb{C}^{}\) identificado con \(\mathbb{R}^{2}\) tenemos que \(\mathbb{C}^{}\) es espacio vectorial real de dimension 2.
La base canónica sería \(\{1,i\}\)

- Pero también como \(\mathbb{C}^{}\) es cuerpo, tenemos que \(\mathbb{C}^{}\) es esapcio vectorial complejo de dimension 1.
La base canónica es \(\{1\}\)

\subsection{Aplicaciones lineales de \(\mathbb{C}^{}\) en \(\mathbb{C}^{}\)}

- Punto de vista real:

%Anotacion 6

Para \(z = x+iy <-> (x,y) \) %Pasar (x,y) a formato vector.

%Anotacion 7.

- Punto de vista complejo:

%Anotacion 8.

\subsection{Rectas y circunferencias en el plano}

Rectas:

- Vision real: \(Ax + By + C = 0\)

Traducimos:

- Vision compleja: \(0 = A \frac{z+z}{2}+B \frac{z-z}{2} + C = \frac{A-iB}{2}z+\frac{a+iB}{2}z_ + C = \beta z + \beta z_ + \gamma \) 

Nos queda entonces:

%Anotacion 9.






