\chapter{Los números complejos}

\section{Los números complejos}
En el siglo XVI se empieza a trabajar con los números complejos, siendo su precursor G. Cardano alrededor de 1545.
Esta idea contó con detractores entre ellos Descartes (1637) y A. De Morgan (1831).
Y por como no podría ser de otra manera al tratarse del tema principal de esta asignatura grandes adeptos como Euler, Gauss, Cauchy, Weierstrass y Riemman.

En 1837 Hamilton define los números complejos, \(\mathbb{C}\), como pares ordenados de números reales, \(\mathbb{R}^{2}=\{(a,b) | a,b \in \mathbb{R}^{}\}\) con las siguientes dos operaciones:

\begin{itemize}
  \item Suma: \((a,b)+(c,d) = (a+c, b+d)\)
  \item Producto: \((a,b)\cdot(c,d) = (ac-bd, ad+bc)\)
\end{itemize}

Estas operaciones son cerradas y con ellas se satisfacen las propiedades de cuerpo:

\begin{enumerate}
  \item La suma es asociativa.
  \item La suma tiene elemento neutro \((0,0)\).
  \item Existe elemento opuesto para la suma. El opuesto de \((a,b)\) es \((-a,-b)\).
  \item La suma es conmutativa.
  \item La multiplicacion es asociativa.
  \item La multiplicacion tiene elemento neutro (distinto del de la suma) \((1,0)\)
  \item Existe elemento inverso para la multiplicación de todo elemnto no nulo. El inverso multiplicativo de \((a,b) \neq (0,0)\) es \((\frac{a}{a^2+b^2},\frac{-b}{a^2+b^2})\)
  \item La multiplicacion es conmutativa.
  \item Propiedad distributiva del producto respecto de la suma.
    \begin{proof}
      \begin{eqnarray*}
        (a,b)((c,d)+(e,f)) & & = (a,b)(c+e,d+f) = \\
                           & & = (ac+ae-bd-bf,ad+af+bc+be)
      \end{eqnarray*}

      \begin{eqnarray*}
        (a,b)(c,d)+(a,b)(e,f) & & = (ac-bd,ad+bc)+(ae-bf,af+be) = \\
                              & & = (ac+ae-bd-bf, ad+bc+af+be)
      \end{eqnarray*}

      \[(a,b)((c,d)+(e,f)) = (a,b)(c,d)+(a,b)(e,f)\]
    \end{proof}
\end{enumerate}

\textbf{Conclusión:} \((\mathbb{R}^{2},+,\cdot)\) es un cuerpo conmutativo al que llamamos \(\mathbb{C}^{}\) y sus elementos se llaman números complejos.

Veamos que los números reales son subcuerpo de \(\mathbb{C}^{}\).

\[ E= \{(a,0) | a \in \mathbb{R}^{}\} \subset \mathbb{C}^{}\]

Las operaciones suma y producto son cerradas en E:

\begin{itemize}
  \item \((a,0) + (b,0) = (a+b, 0) \in E, a,b \in \mathbb{R}^{}\)
  \item \((a,0)\cdot(b,0) = (a\cdot b, 0) \in E, a,b \in \mathbb{R}^{}\)
\end{itemize}

y el opuesto aditivo de \( (a,0) \in E\) es \( (-a,0) \in E\) asi como el inverso multiplicativo de  \( (a,0) \in E\) es \( (\frac{1}{a},0) \in E\).

Esto nos dice que E es subcuerpo de \(\mathbb{C}^{}\). Además, este cuerpo E es isomorfo a \(\mathbb{R}^{}\) (isomorfo en el sentido de cuerpo) mediante la identificación \((a,0) \in E \leftrightarrow a \in \mathbb{R}^{}\)

\section{Terminologia y nomeclatura}

\begin{itemize}
  \item Los elementos de \(\mathbb{C}^{} \leftrightarrow \mathbb{R}^{2}\) se llaman números complejos.
  \item Si \((a,b) \in \mathbb{C}^{} \) su parte real es a y su parte imaginaria es b
  \item Distinguimos dos elementos de \(\mathbb{C}^{}\): \((1,0) = 1\) y \((0,1) = i\)
  \item Mediante la identificación \( E \leftrightarrow \mathbb{R}^{}\). Tenemos que para \(x,y \in \mathbb{R}^{}\).
\[ x\cdot 1=(x,0)(1,0)=(x \cdot 1-0\cdot0, 0\cdot1+x\cdot0)=(x,0)=x \]
\[ y\cdot i=(y,0)(0,1)=(y \cdot 0-0\cdot1, y\cdot1+0\cdot0)=(0,y)=yi \]
Luego \((x,y) = (x,0)+(0,y) = x + yi\) y de esta manera \(\mathbb{C}^{} = \{x+yi : (x,y) \in \mathbb{R}^{2}\}\).
  \item Los números complejos se pueden representar en \(\mathbb{R}^{2}\) como sigue:
Si z es un número complejo, consideramos \(Re(z) = x\) y \(Im(z) = y\).
\end{itemize}

\begin{center} % Grafica sobre la representacion de numeros complejos en el plano.
\begin{tikzpicture}
    \begin{axis}[
        xmin=-2,xmax=2,
        ymin=-2,ymax=2,
        axis x line=middle,
        axis y line=middle,
        axis line style=-,
        xlabel={Re(z)},
        ylabel={Im(z)},
        ]
        \addplot[no marks,red,->] expression[domain=0:1,samples=100]{x} 
                    node[pos=1,anchor=south west]{$x+iy$}; 
    \end{axis}
\end{tikzpicture}
\end{center}

\begin{itemize}
  \item \( \mathbb{C}^{}\) no tiene orden. (A diferencia de \(\mathbb{R}^{}\))
  \item \(i^2 = (1,0)(0,1) = (0 \cdot 0-1 \cdot 1, 0 \cdot 1+1 \cdot 0) = (-1,0) = -1 \).
   
    Esto nos dice que i es solucion de la ecuacion
    \begin{equation*}
      x^2+1=0
    \end{equation*}
    Además otra solución de esta ecuación es -i.

  \item En \(\mathbb{R}^{}\), \(x^2+1=0 \) no tiene raices, pero en \(\mathbb{C}^{}\) si es factorizable como \(x^2+1= (x-i)(x+i)\)
\end{itemize}

Este último punto no aporta otra forma algebraica de introducir \(\mathbb{C}^{}\).

En \(\mathbb{R}[x]\), anillo de los polinomios con coeficientes reales, consideramos el polinomio
\[x^2+1=0\]
este al ser irreducible genera un ideal maximal en  \(\mathbb{R}[x]\).
\[ I = <x^2+1> = \{q(x)(x^2+1), q \in \mathbb{R}[x]\}\]
y por tanto, el cociente, \(\faktor{\mathbb{R}[x]}{I} = \{p(x)+<x^2+1> : p \in \mathbb{R}[x]\}\) es cuerpo bajo una suma y multiplicación inducidas por \(\mathbb{R}^{}[x]\).

Por el algoritmo de la división en \(\mathbb{R}^{}[x]\), todo polinomio p(x) se expresa de forma única como
\[p(x) = q(x)(x^2+1) + a +bx\]
con \(a,b \in \mathbb{R}^{}\)

Esto nos dice que la clase de p(x) en \(\faktor{\mathbb{R}[x]}{I}\) es:
\begin{eqnarray*}
  [p(x)] & = & p(x) + <x^2+1> = [q(x)(x^2+1)+a+bx] = \\
         & = & [q(x)(x^2+1)] + [a+bx] = [a+bx] = [a]+[b][x]
\end{eqnarray*}

Esto nos permite establecer la siguiente identificación:
\begin{eqnarray*}
  \mathbb{C} & \leftrightarrow & \faktor{\mathbb{R}[x]}{I} \\
        a+ib & \leftrightarrow & [a]+[b][x]
\end{eqnarray*}
Tratandose de hecho de un isomorfismo de cuerpos.

Por ejemplo:
\[[x]^2=[x^2+1-1]=[x^2+1]-[1] = [0] - [1] = -[1]\]
\[(i^2 = -1)\]

\begin{definicion}
  Si \(z = a+ib \in \mathbb{C}^{}\), entonces definimos su conjugado como
  \[ \overline{z} = a-ib\]
\end{definicion}

Esta apliación de conjugación se consigue en \(\mathbb{R}^{2}\) con la siguiente aplicación lineal:
\begin{eqnarray*}
  \mathbb{R}^{2} & \longrightarrow & \mathbb{R}^{2} \\
  \begin{pmatrix}
    x\\
    y
  \end{pmatrix} & \longmapsto &
  \begin{pmatrix}
    x\\
    -y
  \end{pmatrix} = \begin{pmatrix}
    1 & 0\\
    0 & -1
  \end{pmatrix} \begin{pmatrix}
    x\\
    y
  \end{pmatrix} 
\end{eqnarray*}

Geometricamente, \(z\) y \(\overline{z}\) son simétricas con respecto al eje real.

\begin{center} % Grafica sobre numeros complejos y sus conjugados.
\begin{tikzpicture}
    \begin{axis}[
        xmin=-2,xmax=2,
        ymin=-2,ymax=2,
        axis x line=middle,
        axis y line=middle,
        axis line style=-,
        xlabel={Re(z)},
        ylabel={Im(z)},
        ]
        \addplot[no marks,red,->] expression[domain=0:1,samples=100]{x} 
                    node[pos=1,anchor=south west]{$z$}; 
        \addplot[no marks,blue,->] expression[domain=0:1,samples=100]{-x} 
                    node[pos=1,anchor=south west]{$\overline{z}$}; 
    \end{axis}
\end{tikzpicture}
\end{center}

\textbf{Propiedades del conjugado:}

\begin{enumerate}
  \item \(0 = \overline{0}\)
  \item \(1 = \overline{1}\)
  \item \(\overline{z+w} = \overline{z}+\overline{w}\)
  \item \(\overline{zw} = \overline{z} \; \overline{w}\)
  \item Se trata de una involucion: \(\overline{\overline{z}} = z\)
  \item Si \(z \in \mathbb{C}^{}\), se tiene que:
    \[ Re(z) = \frac{z+\overline{z}}{2}\]
    \[Im(z) = \frac{z-\overline{z}}{2i}\]
\end{enumerate}

\section{Los complejos como espacio vectorial}

\begin{itemize}
  \item Al estar \(\mathbb{C}\) identificado con \(\mathbb{R}^{2}\) tenemos que \(\mathbb{C}^{}\) es espacio vectorial real de dimensión 2.
  La base canónica sería \(\{1,i\}\)
  \item Pero también como \(\mathbb{C}^{}\) es cuerpo, tenemos que \(\mathbb{C}^{}\) es espacio vectorial complejo de dimensión 1.
  La base canónica es \(\{1\}\)
\end{itemize}

\begin{definicion}[Norma]
  Definimos en los números complejos una norma como sigue:
  \begin{eqnarray*}
    |\cdot| : \mathbb{C}^{ } & \rightarrow & \mathbb{R}^{+} \\
                           z & \mapsto & |z|=\sqrt{(Re(z))^2+(Im(z))^2}
  \end{eqnarray*}
\end{definicion}

Veamos que efectivamente la expresión anterior define una norma:
\begin{enumerate}
  \item \(|z| \geq 0 \)
  \item \( |z| = 0 \Leftrightarrow (Re(z))^2+(Im(z))^2= 0 \Leftrightarrow Re(z) = 0 \wedge Im(z) = 0 \Leftrightarrow z = 0 \) 
  \item Desigualdad triangular: \( |z+w| \leq |z|+|w| \)
    Para provar esto veamos que se satisfacen otras tres propiedades como lema:
    \begin{lema}[3.1]
      \[ Re(z) = x \leq  |z|\]
      \[ Im(z) = y \leq |z|\]
      \begin{proof}
      \[ Re(z) = x \leq |x| = \sqrt{x^2} \leq \sqrt{x^2+y^2} = |z|\]
      \[ Im(z) = y \leq |y| = \sqrt{y^2} \leq \sqrt{x^2+y^2} = |z|\]
      \end{proof}
    \end{lema}
    \begin{lema}[3.2]
      \[|zw| = |z||w|\]
      \begin{proof}
        \begin{eqnarray*}
          |zw| = \sqrt{zw \cdot \overline{zw}} = \sqrt{z \overline{z} \cdot w \overline{w}} = \sqrt{|z|^2|w|^2} = |z||w|  
        \end{eqnarray*}
      \end{proof}
    \end{lema}
    \begin{lema}[3.3]
      \[|\overline{z}| = |z|\]
      \begin{proof}
        \[|\overline{z}| = \sqrt{\overline{z}\; \overline{\overline{z}}}= \sqrt{\overline{z}\; z} \]
      \end{proof}
    \end{lema}
    Demostremos por fin que se satisface la desigualdad triangular:
    \begin{proof}
      \begin{eqnarray*}
        & & |z+w|^2 = (z+w)\overline{(z+w)} = z\overline{z} + w\overline{w} + z\overline{w} + w\overline{z} = |z|^2+|w|^2+2Re(z\overline{w}) \\
        & &  \overset{3.1}{\leq} |z|^2+|w|^2+2|z\overline{w}| \overset{3.2 \;3.3}{=}|z|^2+|w|^2+2|z||w| = \\
        & & = (|z|+|w|)^2 \Rightarrow \\
        & & \Rightarrow |z+w| \leq |z|+|w|
      \end{eqnarray*}
    \end{proof}
  \item Compatibilidad de \( | \cdot | \) con el producto por escalares:
    \[ |\lambda z | = |\lambda| |z| \qquad \text{con} \qquad \lambda \in \mathbb{K}=\mathbb{C},\mathbb{R} \quad z \in \mathbb{C}\]

\end{enumerate}

EL hecho de tener definida la multiplicación en \(\mathbb{C}^{}\), junto con la propiedad 3.2 \(|zw| = |z||w|\) nos dice que \(\mathbb{C}^{}\) es un álgebra (real o compleja) conmutativa (por ser la multiplicación conmutativa).

La norma de \(\mathbb{C}\) proviene de un producto escalar complejo:
\begin{eqnarray*}
  <z,w>_{\mathbb{C}} : \mathbb{C} \times \mathbb{C} & \rightarrow & \mathbb{C} \\
  (z,w) & \mapsto & <z,w>_{\mathbb{C}} = z \overline{w}
\end{eqnarray*}

\textbf{Propiedades:}
\begin{enumerate}
  \item Sesquilial, es decir, lineal respecto a la primera variable y lienal-conjugada respecto de la segunda. 
    \[\lambda_1,\lambda_2,z,z_1,z_2,w,w_1,w_2\in \mathbb{C}\]
    \begin{itemize}
      \item \(<\lambda_1z_1+\lambda_2z_2,w> = (\lambda_1z_1+\lambda_2z_2)\overline{w} = \lambda_1 <z_1,w> + \lambda_2<z_2,w>\)
      \item \(<z,\lambda_1w_1+\lambda_2w_2> = z\overline{(\lambda_1w_1+\lambda_2w_2)} = \overline{\lambda_1} <z,w_1> + \overline{\lambda_2}<z,w_2>\)
    \end{itemize}
  \item Hermiticidad (Simetría conjugada).
    \[ <z,w> = z\overline{w}= \overline{\overline{z}w}= <\overline{z,w}> \] %TODO: arreglar <\overline{z,w}>
  \item Definida positiva.
    \[<z,z> = z \overline{z} = |z|^2 \geq 0 \qquad z \in \mathbb{C}\]
    Además:
    \[<z,z>=0 \Leftrightarrow |z| = 0 \Leftrightarrow z = 0\]
\end{enumerate}

¿Cómo expresar el producto escalar en \(\mathbb{R}^{2}\) en términos complejos?
Sean:
\[z_1 = x_1+iy_1 \in \mathbb{C} \leftrightarrow (x_1,y_1)\]
\[z_2 = x_2+iy_2 \in \mathbb{C} \leftrightarrow (x_2,y_2)\]
El producto escalar real es:
\begin{eqnarray*}
  <(x_1,y_1),(x_2,y_2)>_{\mathbb{R}} = x_1x_2+y_1y_2 = Re(z\overline{z}) = Re(<z_1,z_2>_{\mathbb{C}})
\end{eqnarray*}
\subsection{Aplicaciones lineales}

- Punto de vista real:
\begin{eqnarray*}
  L : \quad \mathbb{C}^{ } & \rightarrow & \mathbb{C}^{} \\
 \begin{pmatrix}
   x \\
   y
 \end{pmatrix} & \mapsto & L\begin{pmatrix}
   x \\
   y
 \end{pmatrix} = \begin{pmatrix}
   L1 | L2
 \end{pmatrix}\begin{pmatrix}
   x \\
   y
 \end{pmatrix}
\end{eqnarray*}

Para \(z = x+iy \leftrightarrow \begin{pmatrix}
 x \\
 y
\end{pmatrix} \). Tenemos el siguiente desarrollo:
\begin{eqnarray*}
  Lz & \leftrightarrow & L\begin{pmatrix}
    x \\
    y
  \end{pmatrix} = \begin{pmatrix}
  a_{11} & a_{12} \\
  a_{21} & a_{22}
  \end{pmatrix}\begin{pmatrix}
    x \\
    y
  \end{pmatrix} = \begin{pmatrix}
  a_{11}x + a_{12}y \\
  a_{21}x + a_{22}y
  \end{pmatrix} \\
   & \leftrightarrow & (a_{11}x + a_{12}y)+i(a_{21}x + a_{22}y) = \\
   & = & (a_{11}+ia_{21})Re(z) + (a_{12}+ia_{22})Im(z) = \\
   & = & (a_{11}+ia_{21})\frac{z+\overline{z}}{2} + (a_{12}+ia_{22})\frac{z-\overline{z}}{2i} = \\
   & = & \dfrac{(a_{11}+a_{22})+i(-a_{12}+a_{21})}{2}z+\dfrac{(a_{11}-a_{22})+i(a_{12}+a_{21})}{2}\overline{z}
\end{eqnarray*}
Si denominamos \( \alpha = \dfrac{(a_{11}+a_{22})+i(-a_{12}+a_{21})}{2} \) y \( \beta = \dfrac{(a_{11}-a_{22})+i(a_{12}+a_{21})}{2} \) tenemos finalmente que la expresión de una aplicación lineal es la que sigue:

\[Lz = \alpha z + \beta \overline{z}\]

- Punto de vista complejo:
\begin{eqnarray*}
  L : \quad \mathbb{C}^{ } & \rightarrow & \mathbb{C}^{} \\
  z \cdot 1 & \mapsto & z \cdot L(1) = \lambda z,  \qquad \lambda \in \mathbb{C}
\end{eqnarray*}

\subsection{Rectas complejas}

Rectas:

- Visión real: \(Ax + By + C = 0\)

Traducimos:

- Visión compleja: \(0 = A \frac{z+z}{2}+B \frac{z-z}{2} + C = \frac{A-iB}{2}z+\frac{a+iB}{2}z_ + C = \beta z + \beta z_ + \gamma \) 

Nos queda entonces:

\[\beta z + \overline{\beta} \; \overline{z} + \gamma = 0, \qquad \beta \in \mathbb{C}-\{0\}, \gamma \in \mathbb{R}^{}\]

\subsection{Circuferencia en los complejos}

\begin{definicion}[Circuferencia en los complejos]
  La circunferencia en \(\mathbb{C}^{}\) de centro \(z_0=x_0+iy_0\) y radio \(r\)
  
  \begin{equation}
    |z-z_0| = r
  \end{equation}

  Manipulando:

\end{definicion}

\begin{eqnarray*} %Anotacion 10
  0 & = |z-z_0|^2 -r^2 = ... = |z|^2+|z_0|^2-2 Re(z\overline{z_0}) - r^2
\end{eqnarray*}

\section{Topologia compleja}
La norma \( \cdot \) en \(\mathbb{C}^{}\), genera una métrica,

\[d(z_1,z_2) = |z_1 - z_2|\]

Una metrica genera una topología.

Las bolas se llaman discos.

\begin{definicion}[Disco abierto]
  Se denomina disco abierto de centro \(z_0\) y radio \(r\):
  \[ \Delta (z_0, r) = D(z_0,r) = \{z \in \mathbb{C}^{} : |z-z_0|< r\}\]
\end{definicion}
\begin{definicion}[Circuferencia]
  Se denomina circunferencia de centro \(z_0\) y radio \(r\):
  \[ \partial \Delta (z_0, r) = C(z_0,r) = \{z \in \mathbb{C}^{} : |z-z_0| = r\}\]
\end{definicion}
\begin{definicion}[Disco cerrado]
  Se denomina disco cerrado de centro \(z_0\) y radio \(r\):
  \[ \overline{\Delta (z_0, r) }= C(z_0,r) = \{z \in \mathbb{C}^{} : |z-z_0| \leq r\}\]
\end{definicion}

\((\mathbb{C}^{}, |\cdot|)\) es un espacio vectorial normado COMPLETO (Toda sucesion de Cauchy converge)

\begin{teorema}[Los números complejos son completos]
  Sea \(\{z_n\}\) una sucesión de números complejos. Si es una sucesión de Cauchy entonces es convergente.
  \begin{proof}
    Sea \(\{z_n\}\) de Cauchy, entonces se tiene que \(\{Re(z_n)\}\) y \(\{Im(z_n)\}\) son sucesiones de Cauchy en \(\mathbb{R}\) y por la completitud de \(\mathbb{R}\) son convergentes a ciertos \(x_0, \;y_0\) respectivamente.

    De aquí se sigue que:
    \[z_n \rightarrow z_0 = x_0+iy_0\]
    Ya que:
    \[|z_n-z_0|^2=(Re(z_n-z_0))^2+(Im(z_n-z_0))^2 \rightarrow 0^2+0^2=0 \]

  \end{proof}
\end{teorema}

En cuanto a la continuidad de funciones:

\begin{eqnarray*}
  & & f:D\subset \mathbb{C} \rightarrow \mathbb{C} \qquad \text{es continua} \Leftrightarrow \\
  \Leftrightarrow & & \begin{cases}
    Re(f): D \rightarrow \mathbb{R} \\
    Im(f): D \rightarrow \mathbb{R}
  \end{cases} \qquad \text{son continuas}
\end{eqnarray*}

\section{Representación polar}
Todo número complejo \(z=x+iy, \; x,y \in \mathbb{R}\) esta identificado con el punto \((x,y)\in \mathbb{R}^{2}\) mediante sus coodenadas cartesianas.

De igual manera tenemos que todo punto de \(\mathbb{R}^{2}\) distinto del origen queda determinado unequivocamente por su módulo y el ángulo que forma con el eje de las abscisas positivas.

%TODO: grafica sobre la representación polar en el plano.

\begin{definicion}[Argumento]

  Si \(z \in \mathbb{C}-\{0\}\) definimos el argumento de z como el siguiente conjunto:
  \[arg(z) = \left\{ \theta \in \mathbb{R} : \cos{\theta}=\frac{Re(z)}{|z|}, \; \sin{\theta}=\frac{Im(z)}{|z|}\right\}\]
\end{definicion}

\begin{proposicion}
 Si \(\theta_0 \in arg(z)\) entonces:
 \[arg(z)= \left\{ \theta_0+2\pi k : k \in \mathbb{Z} \right\}\]
\end{proposicion}
\begin{proof}
  Veamos que \(arg(z)\subset \{\theta_0 +2 \pi k : k \in \mathbb{Z}\}\).

  Sea \(\theta_1 \in arg(z)\). Entonces, como \(\theta_0 \in arg(z)\) se tiene que:
  \begin{eqnarray*}
    \cos(\theta_1) = \frac{Re(z)}{|z|} = \cos(\theta_0) \Rightarrow \theta_1 = \theta_0 + e \pi k, \; k \in \mathbb{Z}
  \end{eqnarray*}
  Por ser la función coseno par se tiene: \(\theta_1 = -\theta_0 + e \pi k, \; k \in \mathbb{Z}\)

  A partir de lo anterior tambien, como también se tiene \( \sin(\theta_1) = \sin(\theta_0)\) tenemos que:
  \begin{eqnarray*}
    \sin(\theta_0) = \sin(-\theta_0+2\pi k ) = 
  \end{eqnarray*}
\end{proof}

\section{Algebra de Banach y función exponencial}
Tenemos entonces que \((\mathbb{C},+,\cdot, |\cdot|)\) es un algebra de Banach conmutativo, luego la teoria de series de potencias tiene perfecto sentido en \(\mathbb{C}\), y en particular podemos definir la exponencial de cualquier número complejo:

Sea \(z=x+iy, \; x,y\in \mathbb{R}\) entonces \(e^z=\sum\limits_{n=0}^{\infty}\frac{z^n}{n!}\)

Además teniendo en cuenta la conmutatividad del producto en \(\mathbb{C}\), tenemos que:
\[e^{z_1+z_2}=e^{z_1}e^{z_2}\]

\subsection{Propiedades de la función exponencial}
\begin{enumerate}
  \item \(e^0=1\)
  \item Sea \(x\in \mathbb{R}\) entonces \(e^x=\sum\limits_{n=0}^{\infty}\frac{x^n}{n!}\) es simplemente la exponencial real, luego con la exponencial compleja nos encontramos ante una extensión de su definición.
  \item Sea \(z=x+iy, \; x,y\in \mathbb{R}\).
    \begin{eqnarray*}
      e^z & = & e^{x+iy}=e^xe^{iy}=e^x\sum\limits_{n=0}^{\infty}\frac{(iy)^n}{n!}= \\
          & = & e^x\left( 1+i\frac{y}{1!}+i^2\frac{y^2}{2!}+i^3\frac{y^3}{3!}+ \cdots \right) = \\
          & = & e^x\left[ \left( 1-\frac{y^2}{2!}+\frac{y^4}{4!} \cdots \right) + i\left( y-\frac{y^3}{3!}+\frac{y^5}{5!} \cdots \right)\right] = \\
          & = & e^x(\cos{y}+i\sin{y})
    \end{eqnarray*}
  \item Sea \(z=x+iy, \; x,y\in \mathbb{R}\) y aplicando la propiedad anterior tenemos:
    \begin{eqnarray*}
      Re(e^z)=e^x\cos{y}=e^{Re(z)}\cos(Im(z)) \\
      Im(e^z)=e^x\sin{y}=e^{Re(z)}\sin(Im(z))
    \end{eqnarray*} 
  \item Como consecuencia del punto anterior tenemos que las funciones \(Re(z)\) y \(Im(z)\) son continuas y por consiguiente \(e^z\) es una función continua en \(\mathbb{C}\).
  \item Sea \(z=x+iy, \; x,y\in \mathbb{R}\).
    \begin{eqnarray*}
      |e^z| & = & \sqrt{(Re(e^z))^2+(Im(e^z))^2} =  \\
            & = & \sqrt{(e^x\cos{y})^2+((e^x\sin{y}))} = \\
            & = & e^x \sqrt{(\cos^2{y}\sin^2{y})} = \\
            & = & e^x = e^{Re(z)}
    \end{eqnarray*} 
  \item Como caso partircular de lo anterior tenemos que:
    \[ |e^{i\theta} | = 1, \qquad \forall \theta \in \mathbb{R} \]
  \item Sea \(z=x+iy, \; x,y\in \mathbb{R}\).
    \begin{eqnarray*}
      \overline{e^z} & = & \overline{e^{x+iy}} = \overline{e^x(\cos{y}+i\sin{y})} = e^x(\cos{y}-i\sin{y}) = \\
                     & = & e^x(\cos{-y}+i\sin{-y}) = e^{x-iy} = e^{\overline{z}}
    \end{eqnarray*}
  \item La exponencial compleja es periodica de periodo \(2 \pi i \). Sea \(z=x+iy, \; x,y\in \mathbb{R}\).
    \begin{eqnarray*}
      e^{z+2\pi i} = e^{z+i(y+2\pi)} = e^x(\cos(y+2\pi)+i\sin(y+2\pi)) = e^x(\cos{y}+i\sin{y}) = e^z
    \end{eqnarray*}
  \item Como consecuencia directa del punto anterior tenemos que la exponencial compleja no es inyectiva.
  \item Veamos que además no es sobreyectiva.
    \[e^z \neq 0 \qquad \forall z \in \mathbb{C}\]
    Ya que \(e^z*e^{-z} = e^{z-z} = e^0 = 1\), luego como la imagen de todos los números complejos es invertible dichas imagenes son no nulas.
  \item De hecho, 0 es el único punto omitido por la exponencial. Es decir, si \(w\in \mathbb{C}-\{0\}\), entonces existe \(z\in \mathbb{C}\) tal que \(e^z=w\).
    \begin{proof}
      Queda pendiente tras la forma polar %TODO
    \end{proof}

\end{enumerate}

