\chapter{Los números complejos}

\section{Los números complejos}
En el siglo XVI se empieza a trabajar con los números complejos, siendo su precursor G. Cardano alrededor de 1545.
Esta idea contó con detractores entre ellos Descartes (1637) y A. De Morgan (1831).
Y por como no podría ser de otra manera al tratarse del tema principal de esta asignatura grandes adeptos como Euler, Gauss, Cauchy, Weierstrass y Riemman.

En 1837 Hamilton define los números complejos,\(\mathbb{C}\), como pares ordenados de números reales, \(\mathbb{R}^{2}=\{(a,b) | a,b \in \mathbb{R}^{}\}\) con las siguientes dos operaciones:

\begin{itemize}
  \item Suma: \((a,b)+(c,d) = (a+c, b+d)\)
  \item Producto: \((a,b)*(c,d) = (ac-bd, ad+bc)\)
\end{itemize}

Estas operaciones son cerradas y con ellas se satisfacen las propiedades de cuerpo:

\begin{enumerate}
  \item La suma es asociativa.
  \item La suma tiene elemento neutro \((0,0)\).
  \item Existe elemento opuesto para la suma. El opuesto de \((a,b)\) es \((-a,-b)\).
  \item La suma es conmutativa.
  \item La multiplicacion es asociativa.
  \item La multiplicacion tiene elemento neutro (distinto del de la suma) \((1,0)\)
  \item Existe elemento inverso para la multiplicación de todo elemnto no nulo. El inverso multiplicativo de \((a,b) \neq (0,0)\) es \((\frac{a}{a^2+b^2},\frac{-b}{a^2+b^2})\)
  \item La multiplicacion es conmutativa.
  \item Propiedad distributiva del producto respecto de la suma.

\begin{eqnarray*}
  (a,b)((c,d)+(e,f)) & & = (a,b)(c+e,d+f) = \\
                     & & = (ac+ae-bd-bf,ad+af+bc+be)
\end{eqnarray*}

\begin{eqnarray*}
  (a,b)(c,d)+(a,b)(e,f) & & = (ac-bd,ad+bc)+(ae-bf,af+be) = \\
                        & & = (ac+ae-bd-bf, ad+bc+af+be)
\end{eqnarray*}

\[(a,b)((c,d)+(e,f)) = (a,b)(c,d)+(a,b)(e,f)\]

\(\hfill \blacksquare\)

\end{enumerate}


\textbf{Conclusión:} \((\mathbb{R}^{2},+,*)\) es un cuerpo conmutativo al que llamamos \(\mathbb{C}^{}\) y sus elementos se llaman números complejos.

Veamos que los números reales son subcuerpo de \(\mathbb{C}^{}\).

\[ E= \{(a,0) | a \in \mathbb{R}^{}\} \subset \mathbb{C}^{}\]

Las operaciones suma y producto son cerradas en E:

\begin{itemize}
  \item \((a,0) + (b,0) = (a+b, 0) \in E, a,b \in \mathbb{R}^{}\)
  \item \((a,0)*(b,0) = (a*b, 0) \in E, a,b \in \mathbb{R}^{}\)
\end{itemize}

y el opuesto aditivo de \( (a,0) \in E\) es \( (-a,0) \in E\) asi como el inverso multiplicativo de  \( (a,0) \in E\) es \( (\frac{1}{a},0) \in E\).

Esto nos dice que E es subcuerpo de \(\mathbb{C}^{}\). Además, este cuerpo E es isomorfo a \(\mathbb{R}^{}\) (isomorfo en el sentido de cuerpo) mediante la identificacion \((a,0) \in E <-> a \in \mathbb{R}^{}\) % Falta cambiar flecha.

\section{Terminologia y nomeclatura}

\begin{itemize}
  \item Los elementos de \(\mathbb{C}^{} <-> \mathbb{R}^{2}\) se llaman números complejos.

    \item Si \((a,b) \in \mathbb{C}^{} \) su parte real es a y su parte imaginaria es b

      \item Distinguimos dos elemntos de \(\mathbb{C}^{}\): \((1,0) = 1\) y \((0,1) = i\)

        \item Mediante la identificacion \( E <-> \mathbb{R}^{}\). Tenemos que para \(x,y \in \mathbb{R}^{}\).
\[ x*1=(x,0)(1,0)=(x*1-0*0, 0*1+x*0)=(x,0)=x \]
\[ y*i=(y,0)(0,1)=(y*0-0*1, y*1+0*0)=(0,y)=yi \]
Luego \((x,y) = (x,0)+(0,y) = x + yi\) y de esta manera \(\mathbb{C}^{} = \{x+yi : (x,y) \in \mathbb{R}^{2}\}\).

  \item Los números complejos se pueden representar en \(\mathbb{R}^{2}\) como sigue:
Si z es un número complejo, consideramos \(Re(z) = x\) y \(Im(z) = y\).
\end{itemize}

% Grafica sobre la representacion de numeros complejos en el plano.
\begin{tikzpicture}[>=stealth]
    \begin{axis}[
        xmin=-4,xmax=4,
        ymin=-2,ymax=2,
        axis x line=middle,
        axis y line=middle,
        axis line style=<->,
        xlabel={$x$},
        ylabel={$y$},
        ]
        \addplot[no marks,blue,<->] expression[domain=0:1,samples=100]{x} 
                    node[pos=0.65,anchor=south west]{$x+iy$}; 
    \end{axis}
\end{tikzpicture}

- \( \mathbb{C}^{}\) no tiene orden. (A diferencia de \(\mathbb{R}^{}\))

- \(i^2 = (1,0)(0,1) = (0*0-1*1, 0*1+1*0)\).

Eston nos dice que i es solucion de la ecuacion
\begin{equation*}
  x^2+1=0
\end{equation*}
Otra solucion es -i.

- En \(\mathbb{R}^{}\), \(x^2+1=0 \) no tiene raices, pero en \(\mathbb{C}^{}\) si es factorizable como \(x^2+1= (x-i)(x+i)\)

- Lo anterior da otra forma algebraica de introducir \(\mathbb{C}^{}\)

En \(\mathbb{R}[x]\), anillo de los polinomios con coeficientes reales, consideramos el polinomio \(x^2+1=0\), el cual, al ser irreducible genera un ideal maximal en  \(\mathbb{R}[x]\).

\[ I = <x^2+1> = \{q(x)(x^2+1), q \in \mathbb{R}[x]\}\]

y por tanto, el cociente, \(\frac{\mathbb{R}[x]}{I} = \{p(x)+<x^2+1> : p \in \mathbb{R}[x]\}\) es cuerpo bajo una suma y multiplicación inducidas por \(\mathbb{R}^{}[x]\).

Por el algoritmo de la división en \(\mathbb{R}^{}[x]\), todo polinomio p(x) se expresa de forma única como

\[p(x) = q(x)(x^2+1) + a +bx\]

con \(a,b \in \mathbb{R}^{}\)

Esto nos dice que la clase de p(x) en \(\frac{\mathbb{R}[x]}{I}\) es:

%Anotacion 2

Esto nos permite establecer la siguiente identificación:

%Anotacion 3.

Por ejemplo:
\[[x]^2=[x^2+1-1]=[x^2+1]-[1] = [0] - [1] = -[1]\]
\[(i^2 = -1)\]

\begin{definicion}
  Si \(z = a+ib \in \mathbb{C}^{}\), entonces definimos su conjugado como
  \[ z = a-ib\] %Falta poner la barra sobre z para indicar el conjugado

  Esta apliación de conjugación se consigue en \(\mathbb{R}^{2}\) con la siguiente aplicación lineal:

  %Anotacion 4

  Geometricamente, z y z son simétricas con respecto al eje real.

  %Grafica 2.

  Propiedades:

  1. 0 = 0
  2. 1 = 1
  3. z+w = z+w
  4. z*w = z*w % Demostracion de esto en anotacion 5.
  5. Se trata de una involucion: z (dos veces conjugado ) = z
  6. Si \(z = a+ib \in \mathbb{C}^{}\), se tiene que
  \[ Re(z) = \frac{z+z}{2}\]
  \[Im(z) = \frac{z-z}{2}\]
\end{definicion}

\section{Los complejos como espacio vectorial}

-Al estar \(\mathbb{C}\) identificado con \(\mathbb{R}^{2}\) tenemos que \(\mathbb{C}^{}\) es espacio vectorial real de dimensión 2.
La base canónica sería \(\{1,i\}\)

- Pero también como \(\mathbb{C}^{}\) es cuerpo, tenemos que \(\mathbb{C}^{}\) es espacio vectorial complejo de dimensión 1.
La base canónica es \(\{1\}\)

\subsection{Aplicaciones lineales}

- Punto de vista real:

%Anotacion 6

Para \(z = x+iy <-> (x,y) \) %Pasar (x,y) a formato vector.

%Anotacion 7.

- Punto de vista complejo:

%Anotacion 8.

\subsection{Rectas y circunferencias en el plano}

Rectas:

- Visión real: \(Ax + By + C = 0\)

Traducimos:

- Visión compleja: \(0 = A \frac{z+z}{2}+B \frac{z-z}{2} + C = \frac{A-iB}{2}z+\frac{a+iB}{2}z_ + C = \beta z + \beta z_ + \gamma \) 

Nos queda entonces:

%Anotacion 9.






